\documentclass[dvipdfmx]{article}

%%%%%%% preamble %%%%%%%
%%言語設定。文書の言語を変更するため、`english' を `japanese' に置き換え。
\usepackage[japanese]{babel}

%ページサイズと余白を設定する
%UK/EU 標準サイズでは `letterpaper' を `a4paper' に置き換える。
\usepackage[letterpaper,top=2cm,bottom=2cm,left=3cm,right=3cm,marginparwidth=1.75cm]{geometry}

% Useful packages ----------------------------------------
\usepackage{amsmath}
\usepackage{graphicx}
\usepackage[colorlinks=true, allcolors=blue]{hyperref}
\usepackage{natbib}
% -----------------------------------------

% Packages for Tables generated by R ----
\usepackage{tabularray}
\UseTblrLibrary{booktabs} 
\usepackage{longtable}
\usepackage{array}
\usepackage{multirow}
\usepackage{wrapfig} % if necessary
\usepackage{float}
\usepackage{colortbl} % if necessary
\usepackage{pdflscape}
\usepackage{tabu}
\usepackage{threeparttable}
\usepackage{threeparttablex}
\usepackage[normalem]{ulem}
\usepackage{makecell}
\usepackage{xcolor} % if necessary
\usepackage{siunitx}
% -----------------------------------------
%%%%%%% preamble %%%%%%%


\title{自分の研究タイトル}
\author{田中 鮎夢}

\begin{document}
\maketitle

\begin{abstract}
この論文はXXについて研究する。分析結果からYYがわかった。
\end{abstract}

\section{はじめに}

本研究の目的は、AとBとの関係を明らかにすることである。



\begin{table}
\centering
\begin{tblr}[         %% tabularray outer open
]                     %% tabularray outer close
{                     %% tabularray inner open
colspec={Q[]Q[]Q[]},
column{1}={halign=l,},
column{2}={halign=c,},
column{3}={halign=c,},
hline{12}={1,2,3}{solid, 0.05em, black},
}                     %% tabularray inner close
\toprule
& (1) & (2) \\ \midrule %% TinyTableHeader
(Intercept)     & \num{-33.751}    & \num{-33.751} \\
& (\num{0.350})    & (\num{0.353}) \\
lgdp\_exporter & \num{1.226}      & \num{1.226}   \\
& (\num{0.008})    & (\num{0.008}) \\
lgdp\_importer & \num{0.951}      & \num{0.951}   \\
& (\num{0.008})    & (\num{0.008}) \\
ldist           & \num{-1.374}     & \num{-1.374}  \\
& (\num{0.022})    & (\num{0.020}) \\
comlang\_off   & \num{1.293}      & \num{1.293}   \\
& (\num{0.051})    & (\num{0.050}) \\
Num.Obs.        & \num{19978}      & \num{19978}   \\
R2              & \num{0.642}      & \num{0.642}   \\
R2 Adj.         & \num{0.642}      & \num{0.642}   \\
AIC             & \num{93181.8}    & \num{93181.8} \\
BIC             & \num{93229.2}    & \num{93229.2} \\
Log.Lik.        & \num{-46584.884} &                \\
RMSE            & \num{2.49}       & \num{2.49}    \\
\bottomrule
\end{tblr}
\end{table}



\end{document}
